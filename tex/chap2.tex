\chapter{Implementacja symulatora}

\par{
W poprzednich rozdziałach opisano generalne zasady wg. których powinien funkcjonować symulator środowiska miejskiego przystosowany do współpracy z systemem fuzji danych. Wskazano czym powinien się cechować by dobrze spełniać stawiane przed nim zadania - w szczególności być w stanie zapewnić dane użyteczne z punktu widzenia prowadzenia badań nad systemem fuzji danych.
}
\par{
W tym zostanie omówiona implementacja takiego symulatora by pokać jak w praktyce można spełnić wymaganiastawiane tego rodzaju aplikacjom, jakie problemy mogą zostać napotkane i jak można sobie z nimi radzić a także jakiego rodzaju technologie można zastosować do uzyskania określonych celów.
}
\par{
W dalszej części rozdziału zostana omówione ograniczenia tego rodzaju symulacji ze szczególnym uwzględnieniem ograniczeń wydajnościowych. Autor postara się zwrócić uwagę na największe ograniczenia takich systemów - czyli miejsca potencjalnej optymalizacji sposobu działania systemu jednocześnie zwracając uwagę na miejsca, które choć mogą wydawać się krytyczne z punktu widzenia symulacji przy konieczności współpracy z systemem fuzji danych przestają w ogóle odgrywać rolę z punktu widzenia wydajności.
}

\section[Wymagania][Wymagania]{Wymagania}

\par{
Tak jak i inne systemy informatyczne tak i ten, choć projektowany w ramach pracy inżynierskiej musiał spełniać pewne wymagania wynikające z możliwości jego późniejszego zastosowania do badań naukowych a w szczególności do pracy Pana Macieja Grzybka dotyczącej śledzenia obiektów w systemie fuzji danych (temat pracy: "Implementacja algorytmu sledzenia obiektów w systemie fuzji danych").
}
\par{
Poniżej opisano postawione systemowi wymagania:
\begin{itemize}
	\item System powinien dostarczać informacje z czujników w mieście w formie wstępnie przetworzonej (na potrzeby fuzji informacji[?]).
	\item System powinien posiadać przynajmniej jeden typ czujnika, który dostarcza informacji w formie zawierającej co najmniej informację na temat współrzędnych geograficznych obiektu obserwowanego.
	\item System powinien udostępniać odczyty z czujników przy użyciu uzgodnionego schematu bazy danych dynamicznych (podział na bazę statyczną i dynamiczną wyjaśniony w rozdziale "Projekt bazy danych").
	\item System powinien mieć możliwość dodania implementacji bardziej zaawansowanych rodzajów czujników w szczególności posiadających inne właściwości obserwacji, dostarczające innych informacji czy też dzia    łające z określonym zaszumieniem.
	\item System powinien prezentować symulację w formie graficznej tak, by można było obserwować jej przebieg i konfrontować go organoleptycznie z wynikami dalszej analizy danych wyjściowych symulacji (ułatwi    enie dla wstępnej fazy prowadzenia badań).
	\item System powinien operować na danych dt. środowiska miejskiego pobranych z odpowiednio zdefiniowanej struktury bazy danych (wczytywanie map).
	\item System powinien umożliwiać regulację natężenia ruchu lub pozwalać na łatwe doimplementowanie tego rodzaju funkcjonalności w przyszłości.
	\item System powinien umożliwiać regulację szybkości symulacji względem czasu rzeczywistego (przyśpieszenie, spowolnienie).
	\item System poiwinien pozwalać na czasowe wstrzymanie symulacji.
\end{itemize}
}

\section[Metodyka symulacji][Metodyka symulacji]{Metodyka symulacji}
\par{ Treść... }
\subsection{Skala symulacji}
\subsection{Upływ czasu}

\section[Dobór technologii][Dobór technologii]{Dobór technologii}
\par{ Treść... }
\subsection{Wydajność}
\subsection{Rozszerzalność}
\subsection{Dostępność}

\section[Projekt bazy danych][Projekt bazy danych]{Projekt bazy danych}
\par{ Treść... }

\section[Architektura systemu][Architektura systemu]{Architektura systemu}
\par{ Treść... }
\subsection{Model MVC}
\subsubsection{O modelu}
\subsubsection{Zastosowanie w aplikacji}
\subsection{Wielowątkowość}
\subsubsection{Biblioteka boost::thread}
\subsubsection{Problemy synchronizacji}

\section[Architektura modelu][Architektura modelu]{Architektura modelu}
\par{ Treść... }
\subsection{Obiekty świata}
\subsection{Interakcja między obiektami}
\subsubsection{Wzorzec wizytatora}
\subsection{Upływ czasu}
\subsubsection{Biblioteka boost::chrono}
\subsubsection{Wzorzec obserwatora}
\subsection{Połączenie z bazą danych}
\subsubsection{Biblioteka pqxx}
\subsubsection{ORM}
\subsection{Komunikacja ze światem (API)}
\subsubsection{Metody publiczne}
\subsubsection{Wołania asynchroniczne}
\subsubsection{Obserwacja stanu symulacji}

\section[Architektura widoku][Architektura widoku]{Architektura widoku}
\par{ Treść... }
\subsection{Biblioteka Qt}
\subsubsection{Wymagania biblioteki}
\subsubsection{QGraphicsLibrary}
\subsubsection{Architektura sygnał-slot}
\subsection{Projektowanie interfejsu graficznego}
\subsubsection{Tworzenie własnych widżetów}

\section[Architektura kontrolera][Architektura kontrolera]{Architektura kontrolera}
\par{ Treść... }
\subsection{Odcięcie od biblioteki Qt}
\subsection{Klasy zdarzeń i ich obsługa}
\subsection{Kolejki a architektura sygnał-slot}