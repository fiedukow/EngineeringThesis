\chapter{Podstawy teoretyczne}
\section[Symulacja komputerowa][Symulacja komputerowa]{Symulacja Komputerowa}

\subsection{Wstęp}
\par{
Ludzkość od zarania dziejów stara się analizować otaczający ją świat. Nie bez powodu. Zrozumienie zasad wg. funkcjonuje otaczająca nas rzeczywistość zdaje się znacząco ułatwiać naszą egzystencję co w prostej linii prowadzi do tego, że samo dążenie do zgłębienia prawideł świata da się wytłumaczyć odwołując się bezpośrednio do ewolucji - osobniki lepiej rozumiejące otaczający świat potrafią lepiej się mierzyć z pojawiającymi się w nim przeciwnościami.
}

\par{
Zasadniczo ludzkie badania działają na dwóch płaszczyznach - dążą do przewidywania przyszłości i pozwalają podejmować coraz lepsze reakcje na bieżące zdarzenia.
}

\par{
Przez wiele lat ludzie prowadzili rozliczne badania starające się wyjaśniać naturę świata - z początku nieco chaotycznie (co nie znaczy, że bez znaczących sukcesów) - u zarania nauki w starożytnej Grecji, gdy cała nauka zamknięta była w jedną dziedzinę nazywaną ogólnym mianem Filozofii.
Wraz z rozwojem ludzkości nasze podejście do nauki jako takiej ewoluowało. I tak już w XVII wieku Kartezjusz wysuwał postulaty, że podstawą nauki powinny być pewne abstrakcyjne narzędzia jak matematyka i logika na bazie których buduje się teorię innych dziedzin. Współczenie niewiele odsuneliśmy się od myśli Kartezjusza - podstawą naszych badań zdaje się być matematyka, na którą nakładane są inne dziedziny jak fizyka i chemia, których wypadkową są nauki przyrodnicze.
}

\par{
Warto zwrócić uwagę, że nasza nauka układa się wsosób warstwowy - kolejne warstwy pozwalają nam odcinać się od reguł obowiązujących w małej skali i budować torie dla skali większej. Jest to szczególnie uzasadnione w świetle odkryć XX wieku, jak ogólna teoria względności.
}

\par{
Oczywiście nadal wiele naukowych tez ma charakter czysto empiryczny i często jest to wystarczające dla określonych zastosowań - w końcu po dziś dzień nauka służy ludziom a nie odwrotnie.
}

\par{
Mamy więc doczynienia z pozornym rozłamem nauki - z jednej strony formalizmy pozwalające na precyzyjny, spójny i co najważniejsze - jednoznaczny, opis określonych zjawisk. Z drugiej strony badania empiryczne, na podstawie których często wyszukuje się potencjalnych dróg rozwoju w sposób analityczny. Obie te metody uzupełniają się wzajemnie - niektórych badań praktycznych nie sposób zaplanować bez określonych reguł i znajomości niektórych torii a jednocześnnie niektóre teorie (w zasadzie większość) nie powstały by gdyby nie konkretne obserwacje rzeczywistości.
}

\par{
Symulacja komputerowa, zdaje się być odpowiedzią na drugą metodykę, która jednakowoż ściśle wykorzystuje specyficzny aparat matematyczny. Stosując ją wykorzystujemy bowiem teorie w pewnej skali, by odtworzyć zachowania i obserować ich skutki w innej (zwykle większej), np. symulując ruch cząsteczek cieczy wg. określonych fizyką zasad by obserwować zachowanie cieczy jako całości w określonych warunkach.
}

\par{
Czym jest więc symulacja komputerowa? Symulacja to proces, który pozwala przy użyciu reguł jednej skali obserwować zjawiska w innej. Słowo komputerowa odnosi się do konkretnej realizacji - z wykorzystaniem maszyn cyfrowych.
}

\subsection{Filozofia symulacji komputerowej}
\subsubsection{Postulat Laplace'a}
\subsubsection{Precyzja komputerów}
\subsubsection{Praktyczne znaczenie ograniczeń}
\subsection{Zastosowanie symulacji}

\subsubsection{Przewidywanie zdarzeń}
Podstawowym zdaniem stawianym przed symulacjami komputerowymi zdaje się być przewidywanie różnego rodzaju zdarzeń.

\subsubsection{Badania empiryczne, eksperymenty}
A co będzie gdy tu pchniemy?

\subsubsection{Testowanie i badania}
Czy to się zachowa jak zakładaliśmy?
Czy nasze mechanizmy zadziałają prawidłowo?
CZY WYŚLEDZIMY OBIEKT

\subsection{Dokładność symulacji komputerowej}
\subsubsection{Postulat Laplace'a}
\subsection{Klasyfikacja symulacji komputerowej}
\subsection{Symulacja a rozwiązania analityczne}
Istnieją dwa podejścia do badania zjawisk o charakterze fizycznym.
Pierwszym z podejść jest badanie analityczne z wykorzystaniem teorii fizyki, bogatego aparatu matematycznego, drugim symulowanie danych zjawisk.
Oba podejścia znacząco się od siebie różnią i są wykorzystywane w różnych sytuacjach.

\subsubsection{Sposób opisu}
Precyzyjny opis zjawisk fizycznych w określonej skali jest zwykle związany z określonym modelem matematycznym. Model ten, wynikający z praw obowiązujących w danej skali (
Przewaga symulacji nad rozwiązaniami dokładnymi objawia się przedewszystkim w stopni złożoności obliczeń. I tak jak symulacja sprowadza się zwykle do przeprowadzenia dużej ilości prostych obliczeń (co jest zadaniem doskonałym dla współczesnych architektur sprzętowych przystosowanych do pracy w takim właśnie trybie) tak rozwiązania dokładne wymagają często ogromnych ilości bardzo złożonych obliczeń lub stosowania pewnych uproszczeń, które sprawiają, że w pewien sposób tracimy powód by korzystać z tego rodzaju technik.
Ponadto symulacja pozwala na przeglądanie zachowań w różnych kontekstach z punktu widzenia skali interpretowalenej dla operatora symulacji.

Symulacja a rozwiązanie analityczne.
- Złożoność
- Liczność przeglądania
- Reprezentacja graficzna (interpretacja organoleptyczna).

\subsection{Symulacja dynamiki obiektów}
\subsubsection{Podstawy fizyczne symulacji}
Dynamika bryły sztywnej
Dynamika punktu materialnego
\subsection{Zastosowanie symulacji}
\subsection{Przykłady istniejących symulacji}
\section[Systemy fuzji danych][Systemy fuzji danych]{Systemy fuzji danych}
\section[Specyfika środowiska miejskiego][Specyfika środowiska miejskiego]{Specyfika środowiska miejskiego}
\subsection{Specyfika symulowanych obiektów}
\subsubsection{Obiekty Statyczne}
\subsubsection{Obiekty Dynamiczne}
\subsection{Specyfika obserwacji}
\subsection{Szumy i niedokładności}