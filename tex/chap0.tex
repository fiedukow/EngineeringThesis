\chapter{Wprowadzenie}

\par{
Systemy, które pozwalają na zbieranie informacji z wielu źródeł i ich łączenie w celu uzyskania spójnego obrazu obserwowanych zjawisk nazywane są systemami fuzji danych \cite{jdl}. Tego rodzaju systemy zdają się być kluczowym elementem rozwoju współczesnej informatyki, ponieważ coraz częściej mamy do czynienia z sytuacjami, w których dane rozproszone są dookoła globu, a dopiero zebranie i przeanalizowanie ich całości pozwala zaobserwować pewne zjawiska i wyciągnąć odpowiednie wnioski.
\par{
Tego rodzaju badania wymagają odpowiedniego wsparcia ze strony symulacji komputerowej. Dla ich prowadzenia zdaje się być bowiem kluczowe, stworzenie systemów mogących zasymulować dane o charakterystyce bliskiej danym rzeczywistym a jednocześnie pozwalające odtworzyć je w formie nie zaszumionej by porównać wyniki systemu fuzji z rzeczywistym obrazem symulowanego świata.
}
\par{
Niniejsza praca opisuje sposób podejścia do projektowania tego rodzaju symulatora. Implementowany system jest symulatorem środowiska miejskiego z systemem miejskiego monitoringu – tego rodzaju system zdaje się doskonale pasować do specyfiki systemów rozproszonych, ze względu na różnorodność zbieranych odczytów, problemy związane z synchronizacją czasu i wieloma źródłami danych, przez które należy w tym wypadku rozumieć czujniki systemu monitoringu.
}
\section{Cel pracy}
\par{
Celem pracy jest zaprojektowanie architektury i zaimplementowanie symulatora środowiska miejskiego. W ramach realizacji tego celu należy stworzyć odpowiedni model środowiska miejskiego.
}
\par{
Projektowany symulator powinien stanowić źródło danych testowych dla systemu ich fuzji. Szczegółowe wymagania dla referencyjnej implementacji opisano w rozdziale 3.1.
}