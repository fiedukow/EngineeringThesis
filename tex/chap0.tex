\chapter{Wprowadzenie}

\par{
Systemy, które pozwalają na zbieranie informacje z wielu źródeł i ich łączenie w celu uzyskania spójnego obrazu obserwowanych zjawisk nazywane są systemami fuzji danych. Tego rodzaju systemy zdają się być kluczowym elementem rozwoju współczesnej informatyki, ponieważ coraz częściej mamy do czynienia z sytuacjami, w których dane rozproszone są dookoła globu a dopiero zebranie i przeanalizowanie ich całości pozwala zaobserwować pewne zjawiska i wyciągnąć odpowiednie wnioski.
\par{
Okazuje się bowiem, że w dzisiejszych czasach problemem nie jest samo posiadanie danych co wyłowienie istotnych informacji z morza informacji nieistotnych. I tak jak systemy związane z  mogą radzić sobie z wnioskowaniem na bazie dostarczonych danych, tak samo dostarczenie kompletnych, niezdublowanych i godnych zaufania danych z wielu źródeł może być problematyczne i stało się obiektem wielu badań.
}
\par{
Tego rodzaju badania wymagają odpowiedniego wsparcia ze strony symulacji komputerowej. Dla ich prowadzenia zdaje się być bowiem kluczowe, stworzenie systemów mogących zasymulować dane o charakterze rozproszonym a jednocześnie pozwalające odtworzyć je w formie nie zaszumionej by porównać wyniki systemu fuzji z rzeczywistym obrazem symulowanego świata.
}
\par{
Niniejsza praca opisuje sposób podejścia do projektowania tego rodzaju symulatora, techniki wykorzystane przy jego implementacji jak i zastosowane przy niej technologie. Implementowany system jest symulatorem środowiska miejskiego z systemem miejskiego monitoringu – tego rodzaju system zdaje się doskonale pasować do specyfiki systemów rozproszonych, ze względu na różnorodność zbieranych odczytów, problemy związane z synchronizacją czasu i wieloma źródłami danych, przez które należy w tym wypadku rozumieć czujniki systemu monitoringu.
}
\par{
Autor niniejszej pracy wyraża nadzieję, że praca ta choć bez wątpienia nie wyczerpie tematu projektowania tego rodzaju aplikacji, to może stanowić cenną wskazówkę dla ich projektantów.
}
\section{Cel pracy}
\par{
Celem pracy jest zaprojektowanie architektury pozwalającej na budowanie systemu symulacji środowiska miejskiego, która będzie umożliwiała uwzględnianie elementów isotnych z punktu widzenia fuzji danych.
}
\par{
Celem pracy jest również stworzenie na bazie tej architektury podstawowej, funkcjonalnej implementacji takiej symulacji, mającej stanowić źródło danych  testowych dla systemu ich fuzji. Szczegółowe wymagania dla referencyjnej implementacji opisano w rozdziale 3.1.
}